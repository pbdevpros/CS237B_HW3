%%%%%%%%%%%%%%%%%%%%%%%%%%
%%%%%%%%%%%%%%%%%%%%%%%%%%
%%%%%% HEADER	 %%%%%%%%%
%%%%%%%%%%%%%%%%%%%%%%%%%%
%%%%%%%%%%%%%%%%%%%%%%%%%%
%%%%%%%%%%%%%%%%%%%%%%%%%%


\documentclass{article}

\usepackage{amsmath}
\usepackage{amsthm}
\usepackage{amssymb}
\usepackage{bbm}
\usepackage{fancyhdr}
% \usepackage{listings}
\usepackage{cite}
\usepackage{graphicx}
\usepackage{enumitem}
\usepackage{courier}
\usepackage[pdftex,colorlinks=true, urlcolor = blue]{hyperref}


\oddsidemargin 0in \evensidemargin 0in
\topmargin -0.5in \headheight 0.25in \headsep 0.25in
\textwidth 6.5in \textheight 9in
\parskip 6pt \parindent 0in \footskip 20pt

% set the header up
\fancyhead{}
\fancyhead[L]{Stanford Computer Science}
\fancyhead[R]{Winter 2020}

%%%%%%%%%%%%%%%%%%%%%%%%%%
\renewcommand\headrulewidth{0.4pt}
\setlength\headheight{15pt}

\usepackage{xparse}
\NewDocumentCommand{\codeword}{v}{%
\texttt{\textcolor{blue}{#1}}%
}

\usepackage{xcolor}
\setlength{\parindent}{0in}

\title{CS 237B: Principles of Robot Autonomy II \\ Problem Set X}
\author{Name: Pádraig (Patrick) Basquel      \\ SUID: 06510441}
\date{}

%%%%%%%%%%%%%%%%%%%%%%%%%%
%%%%%%%%%%%%%%%%%%%%%%%%%%
%%%%%% TITLE	 %%%%%%%%%
%%%%%%%%%%%%%%%%%%%%%%%%%%
%%%%%%%%%%%%%%%%%%%%%%%%%%
%%%%%%%%%%%%%%%%%%%%%%%%%%

\title{CS 237B: Principles of Robot Autonomy II \\ Problem Set X}
\author{Name:      \\ SUID:}
\date{}


%%%
%% PICTURE
%\begin{figure}
%\begin{center}
%\includegraphics[width=13cm,keepaspectratio]{""}
%\caption{}
%\label{fig:fignum}
%\end{center}
%\end{figure}
%%%


\begin{document}

\maketitle
\pagestyle{fancy} 

%%%%%%%%%%%%%%%%%%%%%%%%%%
%%%%%%%%%%%%%%%%%%%%%%%%%%
%%%%%% P1		 %%%%%%%%%
%%%%%%%%%%%%%%%%%%%%%%%%%%
%%%%%%%%%%%%%%%%%%%%%%%%%%
%%%%%%%%%%%%%%%%%%%%%%%%%%

\section*{Problem 1}
\begin{enumerate}[label=(\roman*)]
\item Initializing all weights as zero may result in finding local optima, as opposed to using randomized weight selection. Conversely, the results may be more reproducible if using zero weights.
\item Xavier initialization attempts to keep the variance across the weights of each layer to be the same. The purpose is to prevent what is termed "exploding or vanishing gradients". Exploding gradients can be understood as when weights are initialized too large, the output of each layer increases exponentially. This leads to poor accuracy of the network, as the target oscillates around a minima (the cost oscillating around the minimum value). Similarly, when the weights are initialized to small, the network may converge too quickly, as the cost gets exponentially lower ("vanishing").
\item Adam is a a gradient descent optimization algorithm for stochastic objective function, thus it iteratively used a gradient-based step to find optimal values of parameters for minimizing an objective function. Based on the original paper, Adam achieves faster results in finding optimal parameters for linear/logistic regression, neural networks and convolutional neural networks, as compared to other optimization methods such as SGDNesterov (an accelerated form of stochastic gradient descent) and AdaGrad. The advantages of Adam include working on stochastic (non-stationary) objective and sparse gradients (by using an adaptive learning rate per-parameter). A potential disadvantage has been proposed of Adam being poor at generalizing, compared to SGD for example. [TODO add reference]

\end{enumerate}

%%%%%%%%%%%%%%%%%%%%%%%%%%
%%%%%%%%%%%%%%%%%%%%%%%%%%
%%%%%% P2		 %%%%%%%%%
%%%%%%%%%%%%%%%%%%%%%%%%%%
%%%%%%%%%%%%%%%%%%%%%%%%%%
%%%%%%%%%%%%%%%%%%%%%%%%%%

\section*{Problem 2}
\begin{enumerate}[label=(\roman*)]
\item Using the associative property of matrix multiplication (for rearranging the torque m

\end{enumerate}

%%%%%%%%%%%%%%%%%%%%%%%%%%
%%%%%%%%%%%%%%%%%%%%%%%%%%
%%%%%% P3		 %%%%%%%%%
%%%%%%%%%%%%%%%%%%%%%%%%%%
%%%%%%%%%%%%%%%%%%%%%%%%%%
%%%%%%%%%%%%%%%%%%%%%%%%%%

\section*{Problem 3}
\begin{enumerate}[label=(\roman*)]
\item See code.


\end{enumerate}

\end{document}